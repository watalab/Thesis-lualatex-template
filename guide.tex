\documentclass[./Main.tex]{subfiles}
\graphicspath{{./Fig/}}
\begin{document}

\setcounter{chapter}{-1}
\chapter{このテンプレートの説明}
\section{修論の構成}
修論,卒論ともに通常の学術論文と同じように

\begin{enumerate}
    \item 序論(背景,既存の研究のレビューと目的)
    \item 研究手法
    \item 結果と考察
    \item 結論
    \item その他(参考文献,謝辞,付録等。ただし,参考文献は必須)
\end{enumerate}
の順序で書きます。(実際には「その他」という章は普通作りません)それに対応して,このテンプレートではTeXのソースファイルを
\begin{itemize}
    \item introduction.tex
    \item solution.tex
    \item result.tex
    \item conclusion.tex
\end{itemize}
にしています。参考文献は章ごとに出力するようにしています。

TeXの場合,ソースファイルをどう分割しても特に問題ないので,手法や結果がながければ,さらに細かく分割したほうが見やすいです。

\subsection{LaTeXのエンジン}
このテンプレートはLuaLaTeXでコンパイルすることを想定しています。LuaLaTeXは,Unicodeに対応しており,日本語の扱いが楽です。また,最近のTeXエンジンなので,今後も発展していくことが期待されます。TeX Live 2023以降であれば,特に追加のインストールなしで使えます。
LuaLaTeX用のjsbookクラスとして,ltjsbookを使っています。

注意としては,lulatexの処理はplatexの処理よりも遅いです。大きな文書をコンパイルするときには,platexよりも時間がかかることがあります。下記のファイル分割を適宜使いながら,コンパイル時間を短縮してください。

\subsection{ファイル分割の詳細}
通常,ソースファイルを分割する場合,\verb|\include \input|命令を使います。ですが,ここではsubfilesパッケージを使っています。subfilesパッケージを使うと,分割した各ソースファイルを単独でコンパイルできるようになります。例えば,introduction.texを単独でコンパイルすると,introduction.texだけで完結したPDFが生成されます。これにより,部分的にコンパイルして確認したい場合に便利です。

subfilesでファイルを分割する場合,Mainのソースファイルでは,\verb|\subfile{ファイル名.tex}|とします。\verb|\input|等と異なり,拡張子の.texまで必要です。分割したファイル側には,\verb|\documentclass[./Main.tex]{subfiles}|をはじめに書きます。Main.texが親ファイルであることを指定するためです。このテンプレートのファイルを確認してください。分割したファイル単独でコンパイルできるようになります。コード\ref{code_subfile}に分割したTeXのソース例を示します。最後のifなんちゃらは,大本のMain.texをコンパイルしたときには無視されます。

\begin{lstlisting}[caption=subfileで分割したTeXのソース, label=code_subfile]
\documentclass[./Main.tex]{subfiles}
\graphicspath{{./Fig/}}
\begin{document}

% この間に本文を書く

\ifSubfilesClassLoaded{
\printbibliography[title=参考文献]
}{}

\end{document}
\end{lstlisting}

\subsection{参考文献の出力}
参考文献はbiblatexで出力しています。preamble.texを確認してもらえれば分かります。biblatexはbibtexより新しい機能なので,細かい指定が可能です。

biblatexの設定をしているのはpreamble.TeXの
\begin{verbatim}
    \usepackage[style=phys, biblabel=brackets, backend=biber, refsection=chapter]{biblatex}
\end{verbatim}
の部分です。style=physは物理学系の引用スタイル,biblabel=bracketsは参考文献番号を角括弧で囲む指定,backend=biberはbiberという参考文献処理プログラムを使う指定,refsection=chapterは章ごとに参考文献を分ける指定です。refsectionの指定を外せば,文章全体の参考文献リストになります。

bibファイルの処理にはbiberというコマンドを使うので,自分でコンパイルするときには以下の流れになります。MainのファイルがMain.texとして,

\begin{verbatim}
lualatex Main.tex
biber Main
lualatex Main.tex
lualatex Main.tex
\end{verbatim}
の順でコンパイルします。最初のlualatexでauxファイルが生成され,biberで参考文献データが処理され,その後のlualatexで参考文献が出力されます。2回lualatexを回すのは,参照番号などを正しく反映させるためです。


\section{LaTeXの各パッケージと使い方}

\subsection{zebra-goodies}
todoやcommentを本文に挿入するためのパッケージ。
TeXのソースに
\begin{verbatim}
\todo{ここにTODOを書く}
\comment{XXについて要確認}
\end{verbatim}
という感じで,コマンドを入れると,PDFに赤字でTODOやコメントが挿入されます。

原稿が完成して,todoやコメントを一括で消したいときには
\begin{verbatim}
    \usepackage[final]{zebra-goodies}
\end{verbatim}
でコンパイルすれば,todoやcommentが消えます。

\subsection{cleverref}
図番号・表番号を参照するのに通常は\verb|\ref{}|を使いますが,\verb|\cref{}|を使うと,自動的に「図」「表」などの接頭語が付きます。例えば,\verb|\cref{fig:example}|とすると,「図1」のように出力されます。



\ifSubfilesClassLoaded{
\printbibliography[title=参考文献]
}{}

\end{document}