%% biblatex
% 著者名+年号,参考文献を章ごとに表示 refsection=chapter 後方参照backref=ttue
\usepackage[style=numeric,  backend=biber, backref=true, refsection=chapter]{biblatex}
% 参考文献を章ごとに表示 refsection=chapter,後方参照backref=ttue
%\usepackage[style=phys, biblabel=brackets, backend=biber, backref=true, refsection=chapter]{biblatex}
% 参考文献を一括で表示する場合
%\usepackage[style=phys, biblabel=brackets, backend=biber,]{biblatex}

\addbibresource{mybib.bib}
%% biblatex

\usepackage{graphicx}

\usepackage{amsmath,amssymb,amsfonts}
\usepackage{bm}
\usepackage{siunitx}

\usepackage[colorlinks=true, bookmarks=true,
bookmarksnumbered=true, bookmarkstype=toc, linkcolor=blue,
urlcolor=blue, citecolor=blue]{hyperref}

\usepackage[capitalize]{cleveref}

\crefname{equation}{式}{式}% {環境名}{単数形}{複数形} \crefで引くときの表示
\crefname{figure}{図}{図}% {環境名}{単数形}{複数形} \crefで引くときの表示
\crefname{table}{表}{表}% {環境名}{単数形}{複数形} \crefで引くときの表示
\crefname{algorithm}{Algorithm}{Algorithm}

\crefname{chapter}{第}{第}
\creflabelformat{chapter}{#2#1章#3}
\crefname{section}{第}{第}
\creflabelformat{section}{#2#1節#3}
\crefname{subsection}{第}{第}
\creflabelformat{subsection}{#2#1小節#3}

\usepackage{subfiles}
%\usepackage[final]{zebra-goodies}  
\usepackage{zebra-goodies}  

%% tikz
\usepackage{tikz}
\usepackage{tikz-3dplot}
\usetikzlibrary{calc}
\usetikzlibrary{shadings,intersections}
\usepackage{sansmath}
\usetikzlibrary{angles,quotes,calc}
%%%%%%%5

%% listings
\usepackage{listings, jvlisting}
%\lstset{
%language=TeX,
%basicstyle={\ttfamily},
%identifierstyle={\small},
%% commentstyle={\smallitshape},
%keywordstyle={\small\bfseries},
%ndkeywordstyle={\small},
%stringstyle={\small\ttfamily},
%frame={tb},
%breaklines=true,
%upquote=true,
%columns=[l]{fullflexible},
%numbers=none, %% left
%xrightmargin=0ex,
%xleftmargin=3ex,
%numberstyle={\scriptsize},
%stepnumber=1,
%numbersep=1ex,
%lineskip=-0.5ex
%}
\lstset{
    language=TeX, 
    basicstyle=\ttfamily,
    keywordstyle=\color[RGB]{33,74,135}\bfseries,
    stringstyle=\color[RGB]{79,153,5},
    commentstyle=\color[RGB]{143,89,2}\itshape,
    numberstyle=\footnotesize,
    numbers=none,
    stepnumber=1,
    numbersep=15pt,
    backgroundcolor=\color[RGB]{251,251,251},
    frame=single,
    frameround=ffff,
    framesep=5pt,
    rulecolor=\color[RGB]{148,150,152}, 
    breaklines=true,
    breakautoindent=true,
    breakatwhitespace=true,
    breakindent=25pt,
    showspaces=false,
    showstringspaces=false,
    showtabs=false,
    tabsize=2,
    captionpos=b,
    linewidth=\textwidth,
}

\renewcommand{\lstlistingname}{コード} 

%% 文章記述領域のサイズ指定
\setlength{\textheight}{592pt}
\setlength{\textwidth}{430pt}
\setlength{\oddsidemargin}{20pt}

%%%%  設定

\setcounter{secnumdepth}{3}
\setcounter{tocdepth}{2}  

\renewcommand{\maketitle}{%
\begin{titlepage}
 \null \vskip3em% ページ上部の空白t
 \begin{center}

  {\huge 学位論文(修士)}  \vskip 2em

  {\Huge ベイズ推定を用いたゆで卵の加熱時間の最適化に関する研究} \vskip25em
  {\Large 2025年2月} \vskip1.5em
  
    {\Large 福井大学大学院工学研究科\\安全社会基盤工学専攻原子力安全工学コース}  \vskip1.5em
    
  {\huge ホゲ ホゲたろう}
  %   {\LARGE  } \vskip2em
 \end{center}%
\end{titlepage}}

